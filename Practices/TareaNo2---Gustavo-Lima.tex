% Options for packages loaded elsewhere
\PassOptionsToPackage{unicode}{hyperref}
\PassOptionsToPackage{hyphens}{url}
%
\documentclass[
]{article}
\usepackage{lmodern}
\usepackage{amsmath}
\usepackage{ifxetex,ifluatex}
\ifnum 0\ifxetex 1\fi\ifluatex 1\fi=0 % if pdftex
  \usepackage[T1]{fontenc}
  \usepackage[utf8]{inputenc}
  \usepackage{textcomp} % provide euro and other symbols
  \usepackage{amssymb}
\else % if luatex or xetex
  \usepackage{unicode-math}
  \defaultfontfeatures{Scale=MatchLowercase}
  \defaultfontfeatures[\rmfamily]{Ligatures=TeX,Scale=1}
\fi
% Use upquote if available, for straight quotes in verbatim environments
\IfFileExists{upquote.sty}{\usepackage{upquote}}{}
\IfFileExists{microtype.sty}{% use microtype if available
  \usepackage[]{microtype}
  \UseMicrotypeSet[protrusion]{basicmath} % disable protrusion for tt fonts
}{}
\makeatletter
\@ifundefined{KOMAClassName}{% if non-KOMA class
  \IfFileExists{parskip.sty}{%
    \usepackage{parskip}
  }{% else
    \setlength{\parindent}{0pt}
    \setlength{\parskip}{6pt plus 2pt minus 1pt}}
}{% if KOMA class
  \KOMAoptions{parskip=half}}
\makeatother
\usepackage{xcolor}
\IfFileExists{xurl.sty}{\usepackage{xurl}}{} % add URL line breaks if available
\IfFileExists{bookmark.sty}{\usepackage{bookmark}}{\usepackage{hyperref}}
\hypersetup{
  pdftitle={Practice No.1 by Gustavo Lima Dubón},
  hidelinks,
  pdfcreator={LaTeX via pandoc}}
\urlstyle{same} % disable monospaced font for URLs
\usepackage[margin=1in]{geometry}
\usepackage{color}
\usepackage{fancyvrb}
\newcommand{\VerbBar}{|}
\newcommand{\VERB}{\Verb[commandchars=\\\{\}]}
\DefineVerbatimEnvironment{Highlighting}{Verbatim}{commandchars=\\\{\}}
% Add ',fontsize=\small' for more characters per line
\usepackage{framed}
\definecolor{shadecolor}{RGB}{248,248,248}
\newenvironment{Shaded}{\begin{snugshade}}{\end{snugshade}}
\newcommand{\AlertTok}[1]{\textcolor[rgb]{0.94,0.16,0.16}{#1}}
\newcommand{\AnnotationTok}[1]{\textcolor[rgb]{0.56,0.35,0.01}{\textbf{\textit{#1}}}}
\newcommand{\AttributeTok}[1]{\textcolor[rgb]{0.77,0.63,0.00}{#1}}
\newcommand{\BaseNTok}[1]{\textcolor[rgb]{0.00,0.00,0.81}{#1}}
\newcommand{\BuiltInTok}[1]{#1}
\newcommand{\CharTok}[1]{\textcolor[rgb]{0.31,0.60,0.02}{#1}}
\newcommand{\CommentTok}[1]{\textcolor[rgb]{0.56,0.35,0.01}{\textit{#1}}}
\newcommand{\CommentVarTok}[1]{\textcolor[rgb]{0.56,0.35,0.01}{\textbf{\textit{#1}}}}
\newcommand{\ConstantTok}[1]{\textcolor[rgb]{0.00,0.00,0.00}{#1}}
\newcommand{\ControlFlowTok}[1]{\textcolor[rgb]{0.13,0.29,0.53}{\textbf{#1}}}
\newcommand{\DataTypeTok}[1]{\textcolor[rgb]{0.13,0.29,0.53}{#1}}
\newcommand{\DecValTok}[1]{\textcolor[rgb]{0.00,0.00,0.81}{#1}}
\newcommand{\DocumentationTok}[1]{\textcolor[rgb]{0.56,0.35,0.01}{\textbf{\textit{#1}}}}
\newcommand{\ErrorTok}[1]{\textcolor[rgb]{0.64,0.00,0.00}{\textbf{#1}}}
\newcommand{\ExtensionTok}[1]{#1}
\newcommand{\FloatTok}[1]{\textcolor[rgb]{0.00,0.00,0.81}{#1}}
\newcommand{\FunctionTok}[1]{\textcolor[rgb]{0.00,0.00,0.00}{#1}}
\newcommand{\ImportTok}[1]{#1}
\newcommand{\InformationTok}[1]{\textcolor[rgb]{0.56,0.35,0.01}{\textbf{\textit{#1}}}}
\newcommand{\KeywordTok}[1]{\textcolor[rgb]{0.13,0.29,0.53}{\textbf{#1}}}
\newcommand{\NormalTok}[1]{#1}
\newcommand{\OperatorTok}[1]{\textcolor[rgb]{0.81,0.36,0.00}{\textbf{#1}}}
\newcommand{\OtherTok}[1]{\textcolor[rgb]{0.56,0.35,0.01}{#1}}
\newcommand{\PreprocessorTok}[1]{\textcolor[rgb]{0.56,0.35,0.01}{\textit{#1}}}
\newcommand{\RegionMarkerTok}[1]{#1}
\newcommand{\SpecialCharTok}[1]{\textcolor[rgb]{0.00,0.00,0.00}{#1}}
\newcommand{\SpecialStringTok}[1]{\textcolor[rgb]{0.31,0.60,0.02}{#1}}
\newcommand{\StringTok}[1]{\textcolor[rgb]{0.31,0.60,0.02}{#1}}
\newcommand{\VariableTok}[1]{\textcolor[rgb]{0.00,0.00,0.00}{#1}}
\newcommand{\VerbatimStringTok}[1]{\textcolor[rgb]{0.31,0.60,0.02}{#1}}
\newcommand{\WarningTok}[1]{\textcolor[rgb]{0.56,0.35,0.01}{\textbf{\textit{#1}}}}
\usepackage{graphicx}
\makeatletter
\def\maxwidth{\ifdim\Gin@nat@width>\linewidth\linewidth\else\Gin@nat@width\fi}
\def\maxheight{\ifdim\Gin@nat@height>\textheight\textheight\else\Gin@nat@height\fi}
\makeatother
% Scale images if necessary, so that they will not overflow the page
% margins by default, and it is still possible to overwrite the defaults
% using explicit options in \includegraphics[width, height, ...]{}
\setkeys{Gin}{width=\maxwidth,height=\maxheight,keepaspectratio}
% Set default figure placement to htbp
\makeatletter
\def\fps@figure{htbp}
\makeatother
\setlength{\emergencystretch}{3em} % prevent overfull lines
\providecommand{\tightlist}{%
  \setlength{\itemsep}{0pt}\setlength{\parskip}{0pt}}
\setcounter{secnumdepth}{-\maxdimen} % remove section numbering
\ifluatex
  \usepackage{selnolig}  % disable illegal ligatures
\fi

\title{Practice No.1 by Gustavo Lima Dubón}
\usepackage{etoolbox}
\makeatletter
\providecommand{\subtitle}[1]{% add subtitle to \maketitle
  \apptocmd{\@title}{\par {\large #1 \par}}{}{}
}
\makeatother
\subtitle{Functions}
\author{}
\date{\vspace{-2.5em}}

\begin{document}
\maketitle

What is a Function?

If we talk about programming, a function is a piece of code that is
written to carry a specified task, this is in order to execute segments
of code that we want to use repeatedly. It can or it cannot accept a
series of parameters and it can or cannot return one or more values.

¿Qué es una función?

Si hablamos sobre programación, una función es un fragmento de código el
cual ha sido escrito para realizar una actividad específica, esto con el
fin de ejecutar segmentos de código que queremos usar de manera
repetida. Estas pueden o no pueden aceptar una serie de parámetros y
pueden o no pueden retornar uno o más valores.

Function in R

In R we have subprogram, subroutine, function, procedure, method,
memeber function, operator, anonymous function, closure, lambada
expression, block, but for the scope of this practice we will stay with
the function sematics.

In R, a function is define using the construct:

function (arglist)\{body\}

where the code in between the curly braces is the body function.

Función en R

En R tenemos subprogram, subroutine, function, procedure, method,
memeber function, operator, anonymous function, closure, lambada
expression, block, pero para fines de esta práctica nos mantendremos con
la semántica de función (function).

En R, una función se define usando el siguiente constructor:

function(lista de argumentos)\{cuerpo\}

En donde el código dentro de los corchetes es el cuerpo de la función.

Examples/Ejemplos

\begin{Shaded}
\begin{Highlighting}[]
\CommentTok{\# We will define a vector with 5 values / Definiremos un vector con 5 valores}
\NormalTok{cars }\OtherTok{\textless{}{-}} \FunctionTok{c}\NormalTok{(}\DecValTok{1}\NormalTok{, }\DecValTok{3}\NormalTok{, }\DecValTok{6}\NormalTok{, }\DecValTok{4}\NormalTok{, }\DecValTok{9}\NormalTok{)}
\CommentTok{\# Function will plot a vector passed as parameter / La función graficara un vector pasado como parámetro.}
\NormalTok{function.plotCars }\OtherTok{\textless{}{-}} \ControlFlowTok{function}\NormalTok{(x) }
\NormalTok{\{}
  \CommentTok{\# Graph cars using blue points overlayed by a line / Graficamos los carros usando puntos azules con una línea sobrepuesta.}
  \FunctionTok{plot}\NormalTok{(x, }\AttributeTok{type=}\StringTok{"o"}\NormalTok{, }\AttributeTok{col=}\StringTok{"blue"}\NormalTok{)}
  \CommentTok{\# Create a title with a red, bold/italic font/Creamos un título con fuente roja, negrita y cursiva.}
  \FunctionTok{title}\NormalTok{(}\AttributeTok{main=}\StringTok{"Autos"}\NormalTok{, }\AttributeTok{col.main=}\StringTok{"red"}\NormalTok{, }\AttributeTok{font.main=}\DecValTok{4}\NormalTok{)}
\NormalTok{\}}
\CommentTok{\#We call our function / Llamamos a nuestra funcion}
\FunctionTok{function.plotCars}\NormalTok{(cars)}
\end{Highlighting}
\end{Shaded}

\includegraphics{TareaNo2---Gustavo-Lima_files/figure-latex/unnamed-chunk-1-1.pdf}

Function arguments

There are 2 types of arguments formal arguments and the actual arguments
of a function. The formal arguments are a property of the function,
whereas the actual or calling arguments can vary each time you call a
function. For this practice we will discuss how calling arguments are
mapped to formal arguments, how to call a function given a list of
arguments, how default arguments work.

Argumentos de una función

Hay 2 tipos de argumentos los formales y los argumentos actuales de una
función. Los argumentos formales son propios de la función, mientras que
el actual o el argumento de llamada puede variar cada vez que se llama
una función. Para esta práctica discutiremos como los argumentos de
llamada son mapeados a argumentos formales, como llamar una función dada
una lista de argumentos, como funcionan los argumentos por defecto.

Calling functions

When calling a function we can specify the arguments by position, by
complete name or by partial name. Arguments are matched first by exact
name, then by prefix and finally by position.

\begin{Shaded}
\begin{Highlighting}[]
\NormalTok{function.arguments }\OtherTok{\textless{}{-}} \ControlFlowTok{function}\NormalTok{(a\_arg1, b\_arg2, c\_arg3) \{}
  \FunctionTok{list}\NormalTok{(}\AttributeTok{a =}\NormalTok{ a\_arg1, }\AttributeTok{b =}\NormalTok{ b\_arg2, }\AttributeTok{c =}\NormalTok{ c\_arg3)}
\NormalTok{\}}
\CommentTok{\#Argument in order without specifying the name / Argumentos en orden sin especificar el nombre}
\FunctionTok{str}\NormalTok{(}\FunctionTok{function.arguments}\NormalTok{(}\DecValTok{1}\NormalTok{, }\DecValTok{2}\NormalTok{, }\DecValTok{3}\NormalTok{))}
\end{Highlighting}
\end{Shaded}

\begin{verbatim}
## List of 3
##  $ a: num 1
##  $ b: num 2
##  $ c: num 3
\end{verbatim}

\begin{Shaded}
\begin{Highlighting}[]
\CommentTok{\#Arguments will first match the exact name then match the rest in order / Los argumentos primero se emparejan por nombre exacto y despues el resto por orden.}
\FunctionTok{str}\NormalTok{(}\FunctionTok{function.arguments}\NormalTok{(}\DecValTok{2}\NormalTok{, }\DecValTok{3}\NormalTok{, }\AttributeTok{a\_arg1 =} \DecValTok{1}\NormalTok{))}
\end{Highlighting}
\end{Shaded}

\begin{verbatim}
## List of 3
##  $ a: num 1
##  $ b: num 2
##  $ c: num 3
\end{verbatim}

\begin{Shaded}
\begin{Highlighting}[]
\CommentTok{\#Arguments can be matched by abbreviation as long as its not ambiguous / Los argumentos se pueden emparejar por abreviatura siempre y cuanto no sean ambiguos.}
\FunctionTok{str}\NormalTok{(}\FunctionTok{function.arguments}\NormalTok{(}\DecValTok{2}\NormalTok{, }\DecValTok{3}\NormalTok{, }\AttributeTok{a =} \DecValTok{1}\NormalTok{))}
\end{Highlighting}
\end{Shaded}

\begin{verbatim}
## List of 3
##  $ a: num 1
##  $ b: num 2
##  $ c: num 3
\end{verbatim}

Calling a function given a list of arguments / Llamando a una funcion
dada una lista de argumentos

Suppose you had a list of function arguments / Supongamos que tenemos
una lista de argumentos de funcion:

\begin{Shaded}
\begin{Highlighting}[]
\CommentTok{\#we use na.rm=TRUE to ignore nulls / usamos na.rm=TRUE para ignorar valores null}
\NormalTok{args }\OtherTok{\textless{}{-}} \FunctionTok{list}\NormalTok{(}\DecValTok{1}\SpecialCharTok{:}\DecValTok{10}\NormalTok{, }\AttributeTok{na.rm =} \ConstantTok{TRUE}\NormalTok{)}
\end{Highlighting}
\end{Shaded}

How could you then send that list to mean()? You need do.call() / ¿Como
enviamos esa lista a mean()? Neceistamos do.call():

\begin{Shaded}
\begin{Highlighting}[]
\CommentTok{\#we call the function mean and pass the arguments args / llamamos la funcion mean y pasamos los argumentos args}
\FunctionTok{do.call}\NormalTok{(mean, args)}
\end{Highlighting}
\end{Shaded}

\begin{verbatim}
## [1] 5.5
\end{verbatim}

\begin{Shaded}
\begin{Highlighting}[]
\CommentTok{\# Equivalent to / Equivalente a}
\FunctionTok{mean}\NormalTok{(}\DecValTok{1}\SpecialCharTok{:}\DecValTok{10}\NormalTok{, }\AttributeTok{na.rm =} \ConstantTok{TRUE}\NormalTok{)}
\end{Highlighting}
\end{Shaded}

\begin{verbatim}
## [1] 5.5
\end{verbatim}

Default arguments:

Function arguments in R can have default values.

Argumentos por defecto: Los argumentos de función en R pueden tener
valores por defecto.

\begin{Shaded}
\begin{Highlighting}[]
\NormalTok{function.default }\OtherTok{\textless{}{-}} \ControlFlowTok{function}\NormalTok{(}\AttributeTok{a =} \DecValTok{1}\NormalTok{, }\AttributeTok{b =} \DecValTok{2}\NormalTok{) \{}
  \FunctionTok{c}\NormalTok{(a, b)}
\NormalTok{\}}
\FunctionTok{function.default}\NormalTok{()}
\end{Highlighting}
\end{Shaded}

\begin{verbatim}
## [1] 1 2
\end{verbatim}

\begin{Shaded}
\begin{Highlighting}[]
\NormalTok{function.default2 }\OtherTok{\textless{}{-}} \ControlFlowTok{function}\NormalTok{(}\AttributeTok{a =} \DecValTok{1}\NormalTok{, }\AttributeTok{b =}\NormalTok{ a }\SpecialCharTok{*} \DecValTok{2}\NormalTok{) \{}
  \FunctionTok{c}\NormalTok{(a, b)}
\NormalTok{\}}
\FunctionTok{function.default2}\NormalTok{()}
\end{Highlighting}
\end{Shaded}

\begin{verbatim}
## [1] 1 2
\end{verbatim}

\begin{Shaded}
\begin{Highlighting}[]
\FunctionTok{function.default2}\NormalTok{(}\DecValTok{10}\NormalTok{)}
\end{Highlighting}
\end{Shaded}

\begin{verbatim}
## [1] 10 20
\end{verbatim}

Apply Family

The apply family of functions are sometimes useful as an alternative to
for loops: apply() sapply() vapply()

La familia Apply La familia de funciones apply son a veces utiles como
alternativa para ciclos for: lapply() sapply() vapply()

lapply() Function

The lapply() function is closely related to lists. The lapply() function
is a superior order function, this function applies a function to each
element of a list or vector.

Funcion lapply() La función lapply está estrechamente relacionada con
las listas. La función lapply es una función de orden superior, que
aplica una función a cada elemento de una lista o vector.

Example/Ejemplo

\begin{Shaded}
\begin{Highlighting}[]
\NormalTok{function.lapplyTest }\OtherTok{\textless{}{-}} \ControlFlowTok{function}\NormalTok{(datasetTest)\{}
  \CommentTok{\#We apply the mean function to each element of the list / Aplicamos la funcion mean a cada elemento de la lista}
  \FunctionTok{lapply}\NormalTok{(datasetTest,mean,}\AttributeTok{na.rm =} \ConstantTok{TRUE}\NormalTok{)}
\NormalTok{\}}
\CommentTok{\#We will pass the airquality dataset as parameter from R library / Pasaremos como parametro el dataset airquality de la libreria de R}
\FunctionTok{function.lapplyTest}\NormalTok{(airquality[,}\DecValTok{1}\SpecialCharTok{:}\DecValTok{4}\NormalTok{])}
\end{Highlighting}
\end{Shaded}

\begin{verbatim}
## $Ozone
## [1] 42.12931
## 
## $Solar.R
## [1] 185.9315
## 
## $Wind
## [1] 9.957516
## 
## $Temp
## [1] 77.88235
\end{verbatim}

The lapply() function operates over a list or vector and returns,
always, a list.

La función lapply opera sobre una lista o vector y devuelve, siempre,
una lista.

sapply() Function

The sapply() function is related to the lapply() function, but sapply()
is a simplified version. The sapply() function applies the lapply()
function and analyze the result. If the result can be represented in a
simple manner than a list, it gets simplified.

La funcion sapply() esta relacionada a la funcion lapply(), pero
sapply() es una version simplificada. La funcion sapply() aplica la
funcion lapply() y analiza el resultado. Si este se puede representar de
una manera mas siple que una lista, entonces se simplifica.

Example/Ejemplo

vapply() Function

Whereas sapply() tries to guess the correct format of the result,
vapply() allows to specify it explicit.

Mientras que sapply() trata de adivinar el formato correcto del
resultado, vapply() permite especificarlo explícitamente.

Example/Ejemplo\textless/h2

\begin{Shaded}
\begin{Highlighting}[]
\NormalTok{function.vapplyTest }\OtherTok{\textless{}{-}} \ControlFlowTok{function}\NormalTok{(datasetTest)\{}
  \CommentTok{\#We apply the mean function to each element of the list and specify the type of format / Aplicamos la funcion mean a cada elemento de la lista y especificamos el tipo de formato}
  \FunctionTok{vapply}\NormalTok{(datasetTest, mean,}\AttributeTok{na.rm =} \ConstantTok{TRUE}\NormalTok{, }\FunctionTok{numeric}\NormalTok{(}\DecValTok{1}\NormalTok{))}
\NormalTok{\}}
\CommentTok{\#We will pass the airquality dataset as parameter from R library / Pasaremos como parametro el dataset airquality de la libreria de R}
\FunctionTok{function.vapplyTest}\NormalTok{(airquality[,}\DecValTok{1}\SpecialCharTok{:}\DecValTok{4}\NormalTok{])}
\end{Highlighting}
\end{Shaded}

\begin{verbatim}
##      Ozone    Solar.R       Wind       Temp 
##  42.129310 185.931507   9.957516  77.882353
\end{verbatim}

\end{document}
